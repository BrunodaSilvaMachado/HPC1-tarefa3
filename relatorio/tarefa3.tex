\documentclass[]{article}
\usepackage[utf8x]{inputenc}
\usepackage[brazil,english]{babel}
\usepackage{amsmath,amssymb,amsfonts,amstext}
\usepackage{color}

\definecolor{red}{rgb}{1,0,0}
\definecolor{blue}{rgb}{0,0,1}

%opening
\title{Profiling}
\author{Bruno da Silva Machado}

\begin{document}

\maketitle

\begin{abstract}

Neste artigo vamos efetuar o profiling de dois programas escritos na linguagem C. O primeiro é usa o método da relaxação para descrever um potencial elétrico ao redor de um para-raio, enquanto o segundo programa usa o método de Jacobi para a solução de sistemas de equações lineares. Fazer o profiling dos programas é muito importante pois permite reunir informações sobre o comportamento de um programa, isto é o uso dos recursos do mesmo. e não só isso auxilia na otimização do código pois nos mostra quais funções são as mais utilizadas e se é possível otimiza-las.

\end{abstract}

\section{Descrição dos programas}

\paragraph{O primeiro programa} consiste em solucionar o problema do para-raio que consiste em descobrir o comportamento do potencial elétrico ao redor de uma para-raio. Este problema é descrito pela equação de Laplace, esta por sua vez pode ser solucionado pelo método da relaxação. 

O arquivo que possuí o código deste problema é o "relaxacaoPeriodo.c" e a versão otimizada do código é a  "relaxacaoPeridoOtimizado.c". 

Dentro dos arquivos possui as seguintes funções: \textit{\textcolor{red}{void} \textcolor{blue}{contorno()}} responsável de gerar as condições de contorno na matriz de relaxação. O \textit{\textcolor{red}{void} \textcolor{blue}{relaxacao()}} função responsável de fazer a relaxação da matriz é assim solucionar a equação de laplace, \textit{\textcolor{red}{void} \textcolor{blue}{periodo()}} adiciona e faz a manutenção das condições de periodicidade da matriz de relaxação, o \textit{\textcolor{red}{void} \textcolor{blue}{imprime()}} imprime os dados da matriz dentro de uma arquivo de texto. \textit{\textcolor{red}{void} \textcolor{blue}{trs()}} é a função responsável pela condição de parada que consiste em verificar se o traço da matriz é menor que um certo valor $\epsilon$. Por fim a \textit{\textcolor{red}{int} \textcolor{blue}{main()}} é a entrada do programa e é o local onde as demais funções são chamadas.

\paragraph{O segundo programa} consiste em ler um arquivo de texto com um sistema de equações lineares de quantidade de incógnitas arbitraria e soluciona -lo através do método de Jacobi.

O arquivo que possuí o código deste problema é o "metodoJacobi.c" e a versão otimizada do código é a  "metodoJacobiOtimizado.c". 

Dentro dos arquivos possui as seguintes funções: \textit{\textcolor{red}{double} \textcolor{blue}{norm()}} responsável calcular a norma de um vetor de dimensão $N$. O \textit{\textcolor{red}{double} \textcolor{blue}{* metodoJacobi()}} é um algoritmo clássico para resolver sistemas de equações lineares. Raramente utilizado em sistemas lineares de pequenas dimensões, já que o tempo requerido para obter um minimo de precisão ultrapassa o requerido pelas técnicas diretas como a eliminação gaussiana. Contudo para sistemas grandes, com grande porcentagem de entradas de zeros, essa técnica aparece como uma alternativa mais eficiente. \textit{\textcolor{red}{double} \textcolor{blue}{* substituicaoRegressiva()}} usada neste programa para calcular as raízes do sistema de equações, o \textit{\textcolor{red}{int} \textcolor{blue}{ triangularSuperior\_p()}} reduz a matriz para a forma de triangular superior com pivotamento entre as linhas. \textit{\textcolor{red}{double} \textcolor{blue}{**lerMatrizCompleta()}} recebe o nome do arquivo contendo o sistema de equações lineares e carrega para a memoria para ser usada pelo programa, \textit{\textcolor{red}{void} \textcolor{blue}{ imprimeMatrizCompleta()}} e \textit{\textcolor{red}{void} \textcolor{blue}{imprimeRaiz()}} imprime a matriz informada e as soluções dos sistemas lineares respectivamente.  Por fim a \textit{\textcolor{red}{int} \textcolor{blue}{main()}} é a entrada do programa e é o local onde as demais funções são chamadas.

\section{O profiling}

\subsection{Profiling sem otimização}

\subsection{Profiling com otimização de código}

\subsection{Profiling com otimização de flags}
\end{document}
